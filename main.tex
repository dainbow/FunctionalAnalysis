\documentclass[a4paper,12pt]{article}

%%% Работа с русским языком

\usepackage{cmap}					% поиск в PDF
\usepackage{mathtext} 				% русские буквы в формулах
\usepackage[T2A]{fontenc}			% кодировка
\usepackage[utf8]{inputenc}			% кодировка исходного текста
\usepackage[english,russian]{babel}	% локализация и переносы
\usepackage{indentfirst}            % красная строка в первом абзаце
\usepackage[unicode]{hyperref}
\usepackage{epigraph}
\frenchspacing                      % равные пробелы между словами и предложениями

%%% Дополнительная работа с математикой
\usepackage{amsmath,amsfonts,amssymb,amsthm,mathtools} % пакеты AMS
\usepackage{bbm} % Blackboard bold для цифр
\usepackage{icomma}                                    % "Умная" запятая

\renewcommand{\phi}{\ensuremath{\varphi}}
\renewcommand{\kappa}{\ensuremath{\varkappa}}
\renewcommand{\le}{\ensuremath{\leqslant}}
\renewcommand{\leq}{\ensuremath{\leqslant}}
\renewcommand{\ge}{\ensuremath{\geqslant}}
\renewcommand{\geq}{\ensuremath{\geqslant}}
\renewcommand{\emptyset}{\ensuremath{\varnothing}}

\newcommand{\cl}{\text{cl }}
\newcommand{\setint}{\text{int }}
\newcommand\independent{\protect\mathpalette{\protect\independenT}{\perp}}
\def\independenT#1#2{\mathrel{\rlap{$#1#2$}\mkern2mu{#1#2}}}

\theoremstyle{plain}
\newtheorem{theorem}{Теорема}[section]
\newtheorem{lemma}{Лемма}[section]
\newtheorem{proposition}{Утверждение}[section]
\newtheorem*{corollary}{Следствие}
\newtheorem*{exercise}{Упражнение}

\theoremstyle{definition}
\newtheorem{definition}{Определение}[section]
\newtheorem*{note}{Замечание}
\newtheorem*{reminder}{Напоминание}
\newtheorem*{example}{Пример}
\newtheorem*{tasks}{Вопросы и задачи}

\theoremstyle{remark}
\newtheorem*{solution}{Решение}

%%% Оформление страницы
\usepackage{extsizes}     % Возможность сделать 14-й шрифт
\usepackage{geometry}     % Простой способ задавать поля
\usepackage{setspace}     % Интерлиньяж
\usepackage{enumitem}     % Настройка окружений itemize и enumerate
\usepackage{epigraph}     % Эпиграф
\setlist{leftmargin=25pt} % Отступы в itemize и enumerate

\geometry{top=25mm}    % Поля сверху страницы
\geometry{bottom=30mm} % Поля снизу страницы
\geometry{left=20mm}   % Поля слева страницы
\geometry{right=20mm}  % Поля справа страницы

\begin{document}
\tableofcontents
\newpage

\section{Метрические пространства}

\subsection{Определения}
\begin{definition}
  Метрическим пространством называется множества $X$ с функцией $\rho :\: X^2 \to \mathbb{R}$, обладающей следующими свойствами:
  \begin{enumerate}
    \item $\forall x,\, y \in X :\: \rho(x,\,y) \geq 0$, причём $\rho(x,\,y) = 0 \Leftrightarrow x = y$
    \item $\forall x,\, y \in X :\: \rho(x,\,y) = \rho(y,\,x)$
    \item $\forall x,\,y,\,z \in X :\: \rho(x,\, z) = \rho(x,\,y) + \rho(y,\,z)$ (неравенство треугольника).
  \end{enumerate}
  Функция $\rho$ называется метрикой на множестве $X$.
\end{definition}

\begin{definition}
  Топологическим пространством называется множество $X$ с системой $\tau \subseteq 2^X$, обладающей следующими свойствами:
  \begin{enumerate}
    \item $\emptyset,\, X \in \tau$
    \item $\forall G_1,\, G_2 \in \tau :\: G_1 \cap G_2 \in \tau$
    \item $\forall \{G_\alpha\}_{\alpha \in \mathcal{A}} \subset \tau :\: \bigcup_{\alpha \in \mathcal{A}} G_\alpha \in \tau$
  \end{enumerate}
  Система $\tau$ называется топологией на множестве $X$, а элементы системы $\tau$ -- открытыми множествами.
\end{definition}

\begin{definition}
  Пусть $X$ -- метрическое пространство, $Y \subset X$. Подстранством пространства $X$ называется метрическое пространство $Y$ с метрикой, являющейся сужением метрики на $X$.
\end{definition}

\begin{definition}
  Пусть $X$ -- метрическое пространство. Множество $Y \subset X$ называется ограниченным, если выполнено условие $\sup_{x,\, y \in Y}\rho(x,\,y) < +\infty$
\end{definition}

\begin{definition}
  Пусть $X$ -- метрическое пространство, $x \in X,\, r > 0$:
  \begin{itemize}
    \item Открытым шаром называется множество 
    \[
      B(x,\, r) := \{y \in X \:\vert\: \rho(y,\,x) < r\}
    \]
    \item Замкнутым шаром называется множество
    \[
      \overline{B}(x,\,r) := \{y \in X \:\vert\: \rho(y,\,x) \leq r\}
    \]
  \end{itemize}
\end{definition}

\begin{definition}
  Пусть $X$ -- метрическое пространство, $M \subset X$. Точка $x \in X$ называется внутренней точкой множества $M$ , если существует $r > 0$ такое, что $B(x,\, r) \subset M$. Внутренностью множества $M$ называется множество $\text{int }M$ всех его внутренних точек. Множество $M$ называется открытым, если $\text{int }M = M$.
\end{definition}

\begin{definition}
  Пусть $(X,\, \rho)$ -- метрическое пространство, $M \subset X$. Точка $x \in X$ называется точкой прикосновения множества $M$, если для любого $r > 0$ выполнено условие $B(x,\, r) \cap M \neq \emptyset$. Замыканием множества $M$ называется множество $\overline{M}$ всех его точек прикосновения. Множество $M$ называется замкнутым, если $\overline{M} = M$.
\end{definition}

\begin{definition}
  Пусть $X$ -- метрическое пространство. Множество $A \subset X$ называется:
  \begin{itemize}
    \item Плотным в множестве $B \subset X$, если $B \subset \overline{A}$
    \item Всюду плотным, если $X = \overline{A}$
  \end{itemize}
\end{definition}

\begin{definition}
  Метрическое пространство $X$ называется сепарабельным, если в $X$ существует не более чем счётное всюду плотное множество.
\end{definition}

\begin{definition}
  Пусть $X$ -- метрическое пространство. Последовательность $\{x_n\} \subset X$ сходится к точке $x \in X$, если $\rho(x_n,\, x) \to 0$ при $n \to +\infty$. Обозначение: 
  \[
    x_n \to_X x
  \]
\end{definition}

\begin{definition}
  Пусть $X,\, Y$ -- метрические пространства. $f :\: X \to Y$. Отображение $f$ называется непрерывным в точке $x \in X$, если выполнено одно из следующих условий:
  \begin{enumerate}
    \item Для любого $\varepsilon > 0$ существует $\delta > 0$ такое, что $f(B(x,\, \delta)) \subset B(f(x),\, \varepsilon)$
    \item Для любой $\{x_n\} \subset X$ такой, что $x_n \to_X x$, выполнено $f(x_n) \to_Y f(x)$
  \end{enumerate}
\end{definition}

\subsection{Несложные утверждения}

\begin{lemma}
  Неравенство Минсковского.

  Пусть $E$ -- измеримое множество, на котором задана мера $\mu$, и пусть $f,\,g :\: E \to \mathbb{R}$ -- измеримые функции. Тогда выполнено следующее:
  \[
    \left(\int_E \vert f(x) + g(x)\vert^pd\mu\right)^{\frac{1}{p}} \leq \left(\int_E \vert f(x)\vert^p d\mu\right)^{\frac{1}{p}} + \left(\int_E \vert g(x)\vert^pd\mu\right)^{\frac{1}{p}}
  \]
\end{lemma}

\begin{lemma}
  Неравенство Гёльдера.

  Пусть $E$ измеримое множество, на котором задана мера $\mu$. Тогда для любых $p,\,q \geq 1,\, \frac{1}{p} + \frac{1}{q} = 1$, если $f \in L^p(E),\, g \in L^q(E)$, то $f \cdot g \in L^1$, причём выполнено следующее:
  \[
    \int_E \vert f(x)g(x) \vert d\mu \leq \left(\int_E \vert f(x)\vert d\mu\right)^{\frac{1}{p}} \cdot \left(\int_E \vert g(x) \vert d\mu\right)^{\frac{1}{q}}
  \]
\end{lemma}

\begin{lemma}
  Пусть $X$ -- метрическое пространство, $M \subset X$. Тогда множество $M$ открыто $\Leftrightarrow$ множество $X \setminus M$ замкнуто.
\end{lemma}

\begin{proof}
  Достаточно заметить, что
  \[
    x \in \overline{X \setminus M} \Leftrightarrow \forall r > 0 :\: B(x,\, r) \cap (X \setminus M) \neq \emptyset \Leftrightarrow x \not\in \text{int }M
  \]
  Значит, $\text{int }M = M \Leftrightarrow \overline{X \setminus M} = X \setminus M$.
\end{proof}

\begin{lemma}
  Пусть $X$ -- метрическое пространство. Тогда:
  \begin{enumerate}
    \item Для любого $x \in X$ и $r > 0$ множество $B(x,\,r)$ -- открытое.
    \item Для любого $x \in X$ и $r > 0$ множество $\overline{B}(x,\,r)$ -- замкнутое.
    \item Для любого множества $M \subset X$ множество $\text{int }M$ -- открытое, причём наибольшее по включение открытое множество, содержащееся в $M$.
    \item Для любого множества $M \subset X$ множество $\overline{M}$ -- замкнутое, причём наименьшее по включению замкнутое множество, содержащее $M$.
  \end{enumerate}
\end{lemma}

\begin{proof}
  \begin{enumerate}
    \item Пусть $y \in B(x,\,r)$, тогда, по неравенству треугольника, $B(y,\, r - \rho(x,\,y)) \subset B(x,\,r)$, то есть $y \in \text{int }B(x,\,r)$.
    \item Пусть $y \in \overline{\overline{B}(x,\,r)}$, тогда для любого $\varepsilon > 0$ выполнено $B(y,\,\varepsilon) \cap \overline{B}(x,\,r) \neq \emptyset$, откуда, по неравенству треугольника, $\rho(x,\,y) < r + \varepsilon$. В силу произвольности числа $\varepsilon$, получаем, что $\rho(x,\,y) \leq r$, то есть $y \in \overline{B}(x,\,r)$.
    \item Для любого открытого множества $G \subset M$ выполнено $G = \text{int G} \subset \text{int M}$, поэтому, в частности, множество $\text{int M}$ открыто, как объединение всех содержащихся в $M$ открытых множеств.
    \item Для любого замкнутого множества $F \supset M$ выполнено $F = \overline{F} \supset \overline{M}$, поэтому, в частности, множество $\overline{M}$ замкнуто, как пересечение всех содержащих $M$ замкнутых множеств.
  \end{enumerate}
\end{proof}

\begin{lemma}
  Пусть $X,\, Y$ -- метрические пространства, $f :\: X \to Y$. Тогда следующие условия эквивалентны:
  \begin{itemize}
    \item Отображение $f$ непрерывно.
    \item Для любого открытого множества $G \subset Y$ множество $f^{-1}(G)$ тоже является открытм
  \end{itemize}
\end{lemma}

\begin{proof}
  \begin{itemize}
    \item ($1 \Rightarrow 2$) Зафиксируем произвольное открытое множество $G \subset Y$. Тогда, поскольку выполнено равенство $f^{-1}(G) = \bigcup_{y \in G}f^{-1}(y)$ и каждое множество $f^{-1}(y)$ является открытым (из определения непрерывности), множество $f^{-1}(G)$ тоже является открытым.
    \item ($2 \Rightarrow 1$) Зафиксируем произвольные $x \in X,\, \varepsilon >0$. Множество $B(f(x),\, \varepsilon)$ является открытым, поэтому его прообраз тоже открыт, то есть существует $\delta >0$ такое, что $f(B(x,\, \delta)) \subset B(f(x),\, \varepsilon)$, что и даёт требуемое в силу произвольности выбора точки $x$ и числа $\varepsilon$.
  \end{itemize}
\end{proof}

\end{document}