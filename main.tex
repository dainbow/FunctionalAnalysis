\documentclass[a4paper,12pt]{article}

%%% Работа с русским языком

\usepackage{cmap}					% поиск в PDF
\usepackage{mathtext} 				% русские буквы в формулах
\usepackage[T2A]{fontenc}			% кодировка
\usepackage[utf8]{inputenc}			% кодировка исходного текста
\usepackage[english,russian]{babel}	% локализация и переносы
\usepackage{indentfirst}            % красная строка в первом абзаце
\usepackage[unicode]{hyperref}
\usepackage{epigraph}
\frenchspacing                      % равные пробелы между словами и предложениями

%%% Дополнительная работа с математикой
\usepackage{amsmath,amsfonts,amssymb,amsthm,mathtools} % пакеты AMS
\usepackage{bbm} % Blackboard bold для цифр
\usepackage{icomma}                                    % "Умная" запятая

\renewcommand{\phi}{\ensuremath{\varphi}}
\renewcommand{\kappa}{\ensuremath{\varkappa}}
\renewcommand{\le}{\ensuremath{\leqslant}}
\renewcommand{\leq}{\ensuremath{\leqslant}}
\renewcommand{\ge}{\ensuremath{\geqslant}}
\renewcommand{\geq}{\ensuremath{\geqslant}}
\renewcommand{\emptyset}{\ensuremath{\varnothing}}

\newcommand{\cl}{\text{cl }}
\newcommand{\setint}{\text{int }}
\newcommand\independent{\protect\mathpalette{\protect\independenT}{\perp}}
\def\independenT#1#2{\mathrel{\rlap{$#1#2$}\mkern2mu{#1#2}}}

\theoremstyle{plain}
\newtheorem{theorem}{Теорема}[section]
\newtheorem{lemma}{Лемма}[section]
\newtheorem{proposition}{Утверждение}[section]
\newtheorem*{corollary}{Следствие}
\newtheorem*{exercise}{Упражнение}

\theoremstyle{definition}
\newtheorem{definition}{Определение}[section]
\newtheorem*{note}{Замечание}
\newtheorem*{reminder}{Напоминание}
\newtheorem*{example}{Пример}
\newtheorem*{tasks}{Вопросы и задачи}

\theoremstyle{remark}
\newtheorem*{solution}{Решение}

%%% Оформление страницы
\usepackage{extsizes}     % Возможность сделать 14-й шрифт
\usepackage{geometry}     % Простой способ задавать поля
\usepackage{setspace}     % Интерлиньяж
\usepackage{enumitem}     % Настройка окружений itemize и enumerate
\usepackage{epigraph}     % Эпиграф
\setlist{leftmargin=25pt} % Отступы в itemize и enumerate

\geometry{top=25mm}    % Поля сверху страницы
\geometry{bottom=30mm} % Поля снизу страницы
\geometry{left=20mm}   % Поля слева страницы
\geometry{right=20mm}  % Поля справа страницы

\begin{document}
\tableofcontents
\newpage

\section{Метрические пространства}

\subsection{Определения}
\begin{definition}
	Метрическим пространством называется множество $X$ с функцией $\rho :\: X^2 \to \mathbb{R}$, обладающей следующими свойствами:
	\begin{enumerate}
		\item $\forall x,\, y \in X :\: \rho(x,\,y) \geq 0$, причём $\rho(x,\,y) = 0 \Leftrightarrow x = y$
		\item $\forall x,\, y \in X :\: \rho(x,\,y) = \rho(y,\,x)$
		\item $\forall x,\,y,\,z \in X :\: \rho(x,\, z) \leq \rho(x,\,y) + \rho(y,\,z)$ (неравенство треугольника).
	\end{enumerate}
	Функция $\rho$ называется метрикой на множестве $X$.
\end{definition}

\begin{definition}
	Топологическим пространством называется множество $X$ с системой $\tau \subseteq 2^X$, обладающей следующими свойствами:
	\begin{enumerate}
		\item $\emptyset,\, X \in \tau$
		\item $\forall G_1,\, G_2 \in \tau :\: G_1 \cap G_2 \in \tau$
		\item $\forall \{G_\alpha\}_{\alpha \in \mathcal{A}} \subset \tau :\: \bigcup_{\alpha \in \mathcal{A}} G_\alpha \in \tau$
	\end{enumerate}
	Система $\tau$ называется топологией на множестве $X$, а элементы системы $\tau$ -- открытыми множествами.
\end{definition}

\begin{definition}
	Пусть $X$ -- метрическое пространство, $Y \subset X$. Подпространством пространства $X$ называется метрическое пространство $Y$ с метрикой, являющейся сужением метрики на $X$.
\end{definition}

\begin{definition}
	Пусть $X$ -- метрическое пространство. Множество $Y \subset X$ называется ограниченным, если выполнено условие $\sup_{x,\, y \in Y}\rho(x,\,y) < +\infty$
\end{definition}

\begin{definition}
	Пусть $X$ -- метрическое пространство, $x \in X,\, r > 0$:
	\begin{itemize}
		\item Открытым шаром называется множество
		      \[
			      B(x,\, r) := \{y \in X \:\vert\: \rho(y,\,x) < r\}
		      \]
		\item Замкнутым шаром называется множество
		      \[
			      \overline{B}(x,\,r) := \{y \in X \:\vert\: \rho(y,\,x) \leq r\}
		      \]
	\end{itemize}
\end{definition}

\begin{definition}
	Пусть $X$ -- метрическое пространство, $M \subset X$. Точка $x \in X$ называется внутренней точкой множества $M$ , если
	\[
		\exists r > 0 :\: B(x,\, r) \subset M
	\]
	Внутренностью множества $M$ называется множество $\text{int }M$ всех его внутренних точек. Множество $M$ называется открытым, если $\text{int }M = M$.
\end{definition}

\begin{definition}
	Пусть $(X,\, \rho)$ -- метрическое пространство, $M \subset X$. Точка $x \in X$ называется точкой прикосновения множества $M$, если
	\[
		\forall r > 0 :\: B(x,\, r) \cap M \neq \emptyset
	\]
	Замыканием множества $M$ называется множество $\overline{M}$ всех его точек прикосновения. Множество $M$ называется замкнутым, если $\overline{M} = M$.
\end{definition}

\begin{definition}
	Пусть $X$ -- метрическое пространство. Множество $A \subset X$ называется:
	\begin{itemize}
		\item Плотным в множестве $B \subset X$, если $B \subset \overline{A}$
		\item Всюду плотным, если $X = \overline{A}$
	\end{itemize}
\end{definition}

\begin{definition}
	Метрическое пространство $X$ называется сепарабельным, если в $X$ существует не более чем счётное всюду плотное множество.
\end{definition}

\begin{definition}
	Пусть $X$ -- метрическое пространство. Последовательность $\{x_n\}_{n = 1}^\infty \subset X$ сходится к точке $x \in X$, если $\rho(x_n,\, x) \to 0$ при $n \to +\infty$. Обозначение:
	\[
		x_n \to_X x
	\]
\end{definition}

\begin{definition}
	Пусть $X,\, Y$ -- метрические пространства. $f :\: X \to Y$. Отображение $f$ называется непрерывным в точке $x \in X$, если выполнено одно из следующих условий:
	\begin{enumerate}
		\item Для любого $\varepsilon > 0$ существует $\delta > 0$ такое, что $f(B(x,\, \delta)) \subset B(f(x),\, \varepsilon)$
		\item Для любой $\{x_n\}_{n = 1}^\infty \subset X$ такой, что $x_n \to_X x$, выполнено $f(x_n) \to_Y f(x)$
	\end{enumerate}
\end{definition}

\subsection{Несложные утверждения}

\begin{lemma}
	Неравенство Гёльдера.

	Пусть $E$ измеримое множество, на котором задана мера $\mu$. Тогда для любых $p,\,q \geq 1,\, \frac{1}{p} + \frac{1}{q} = 1$, если $f \in L^p(E),\, g \in L^q(E)$, то $f \cdot g \in L^1$, причём выполнено следующее:
	\[
		\int_E \vert f(x)g(x) \vert d\mu \leq \left(\int_E \vert f(x)\vert^p d\mu\right)^{\frac{1}{p}} \cdot \left(\int_E \vert g(x) \vert^q d\mu\right)^{\frac{1}{q}}
	\]
\end{lemma}

\begin{proof}
	Для доказательства воспользуемся неравенством Юнга:
	\[
		a,\,b \geq 0,\, 1 < p < +\infty,\, \frac{1}{p} + \frac{1}{q} = 1 \Rightarrow ab \leq \frac{a^p}{p} + \frac{b^q}{q}
	\]
	Положим
	\[
		A := \left(\int_Ef^pd\mu\right)^{\frac{1}{p}} ;\;\;\;\; B := \left(\int_Eg^qd\mu\right)^{\frac{1}{q}}
	\]
	Если $A = 0 \Rightarrow f \overset{\text{п.в.}}{=} 0 \Rightarrow f\cdot g  \overset{\text{п.в.}}{=} 0 \Rightarrow \int_Ef\cdot gd\mu = 0$.

	Если же $A = \infty,\, B > 0$, то $AB = \infty$ и аналогично остальные случаи с бесконечностями.

	Если $0 < A,\, B < \infty \Rightarrow$ по неравенству Юнга имеем:
	\[
		\frac{1}{AB}\int_Ef\cdot gd\mu = \int_E\frac{f}{A}\frac{g}{B}d\mu \leq \int_E \left(\frac{1}{p}\left(\frac{f}{A}\right)^p + \frac{1}{q}\left(\frac{g}{B}\right)^q\right)d\mu = \frac{1}{p}\frac{\int_Ef^pd\mu}{A^p} + \frac{1}{q}\frac{\int_Eg^qd\mu}{B^q} = \frac{1}{p} + \frac{1}{q} = 1
	\]
\end{proof}

\begin{lemma}
	Неравенство Минсковского.

	Пусть $E$ -- измеримое множество, на котором задана мера $\mu$, и пусть $f,\,g :\: E \to \mathbb{R}$ -- измеримые функции. Тогда выполнено следующее:
	\[
		\left(\int_E \vert f(x) + g(x)\vert^pd\mu\right)^{\frac{1}{p}} \leq \left(\int_E \vert f(x)\vert^p d\mu\right)^{\frac{1}{p}} + \left(\int_E \vert g(x)\vert^pd\mu\right)^{\frac{1}{p}}
	\]
\end{lemma}

\begin{proof}
	Положим
	\[
		A := \left(\int_Ef^pd\mu\right)^{\frac{1}{p}} ;\;\;\;\; B := \left(\int_Eg^pd\mu\right)^{\frac{1}{p}} ;\;\;\;\; C := \left(\int_E(f + g)^pd\mu\right)^{\frac{1}{p}}
	\]
	Будем полагать, что $A,\,B < \infty$. (Иначе неравенство тривиально). Тогда
	\[
		(f + g)^p \leq (2\max\{f,\,g\})^p \leq 2^p(f^p + g^p) \Rightarrow C < \infty
	\]
	Тогда введём $q := \frac{p}{p-1}$ и использум неравенство Гёльдера в следующем виде:
	\begin{align*}
		C^p = \int_E f(f + g)^{p - 1}d\mu + \int_Eg(f + g)^{p-1}d\mu \overset{\text{Гёльдер}}{\leq}                                                                                                                 \\
		\left(\int_E f^pd\mu\right)^{\frac{1}{p}}\cdot\left(\int_E(f + g)^{(p - 1)q}d\mu\right)^{\frac{1}{q}} + \left(\int_Eg^pd\mu\right)^{\frac{1}{p}}\cdot\left(\int_E(f + g)^{(p-1)q}d\mu\right)^{\frac{1}{q}}= \\
		(A + B)\left(\int_E (f + g)^pd\mu\right)^{\frac{1}{q}} = (A + B)C^{p -1}
	\end{align*}
\end{proof}

\begin{lemma}
	Пусть $X$ -- метрическое пространство, $M \subset X$. Тогда множество $M$ открыто $\Leftrightarrow$ множество $X \setminus M$ замкнуто.
\end{lemma}

\begin{proof}
	Достаточно заметить, что
	\[
		x \in \overline{X \setminus M} \Leftrightarrow \forall r > 0 :\: B(x,\, r) \cap (X \setminus M) \neq \emptyset \Leftrightarrow x \not\in \text{int }M
	\]
	Значит, $\text{int }M = M \Leftrightarrow \overline{X \setminus M} = X \setminus M$.
\end{proof}

\begin{lemma}
	Пусть $X$ -- метрическое пространство. Тогда:
	\begin{enumerate}
		\item Для любого $x \in X$ и $r > 0$ множество $B(x,\,r)$ -- открытое.
		\item Для любого $x \in X$ и $r > 0$ множество $\overline{B}(x,\,r)$ -- замкнутое.
		\item Для любого множества $M \subset X$ множество $\text{int }M$ -- открытое, причём наибольшее по включение открытое множество, содержащееся в $M$.
		\item Для любого множества $M \subset X$ множество $\overline{M}$ -- замкнутое, причём наименьшее по включению замкнутое множество, содержащее $M$.
	\end{enumerate}
\end{lemma}

\begin{proof}
	\begin{enumerate}
		\item Пусть $y \in B(x,\,r)$, тогда, по неравенству треугольника, $B(y,\, r - \rho(x,\,y)) \subset B(x,\,r)$, то есть $y \in \text{int }B(x,\,r)$.
		\item Пусть $y \in \overline{\overline{B}(x,\,r)}$, тогда для любого $\varepsilon > 0$ выполнено $B(y,\,\varepsilon) \cap \overline{B}(x,\,r) \neq \emptyset$, откуда, по неравенству треугольника, $\rho(x,\,y) < r + \varepsilon$. В силу произвольности числа $\varepsilon$, получаем, что $\rho(x,\,y) \leq r$, то есть $y \in \overline{B}(x,\,r)$.
		\item Для любого открытого множества $G \subset M$ выполнено $G = \text{int G} \subset \text{int M}$, поэтому, в частности, множество $\text{int M}$ открыто, как объединение всех содержащихся в $M$ открытых множеств.
		\item Для любого замкнутого множества $F \supset M$ выполнено $F = \overline{F} \supset \overline{M}$, поэтому, в частности, множество $\overline{M}$ замкнуто, как пересечение всех содержащих $M$ замкнутых множеств.
	\end{enumerate}
\end{proof}

\begin{lemma}
	Пусть $X,\, Y$ -- метрические пространства, $f :\: X \to Y$. Тогда следующие условия эквивалентны:
	\begin{itemize}
		\item Отображение $f$ непрерывно.
		\item Для любого открытого множества $G \subset Y$ множество $f^{-1}(G)$ тоже является открытм
	\end{itemize}
\end{lemma}

\begin{proof}
	\begin{itemize}
		\item ($1 \Rightarrow 2$) Зафиксируем произвольное открытое множество $G \subset Y$. Тогда, поскольку выполнено равенство $f^{-1}(G) = \bigcup_{y \in G}f^{-1}(y)$ и каждое множество $f^{-1}(y)$ является открытым (из определения непрерывности), множество $f^{-1}(G)$ тоже является открытым.
		\item ($2 \Rightarrow 1$) Зафиксируем произвольные $x \in X,\, \varepsilon >0$. Множество $B(f(x),\, \varepsilon)$ является открытым, поэтому его прообраз тоже открыт, то есть существует $\delta >0$ такое, что $f(B(x,\, \delta)) \subset B(f(x),\, \varepsilon)$, что и даёт требуемое в силу произвольности выбора точки $x$ и числа $\varepsilon$.
	\end{itemize}
\end{proof}

\section{Полные метрические пространства}
\subsection{Теорема о вложенных шарах, теорема Бэра}
\begin{definition}
	Пусть $X$ -- метрическое пространство. Последовательность $\{x_n\}_{n = 1}^\infty \subset X$ называется фундаментальной, если выполнено следующее условие:
	\[
		\forall \varepsilon > 0 \: \exists N \in \mathbb{N} \: \forall n,\,m \geq N :\: \rho(x_n,\, x_m) < \varepsilon
	\]
\end{definition}

\begin{definition}
	Метрическое пространство называется полным, если любая фундаментальная последовательность в нём сходится.
\end{definition}

\begin{theorem}
	О вложенных шарах.

	Пусть $X$ -- полное метрическое пространство. $\{\overline{B}(x_n,\, r_n)\}_{n = 1}^\infty$ -- последовательность вложенных замкнутых шаров такая, что $r_n \to 0$. Тогда $\bigcap_{n = 1}^\infty \overline{B}(x_n,\, r_n) = \{x^*\}$ для некоторой точки $x^* \in X$.
\end{theorem}

\begin{proof}
	В силу вложенности шаров и условия $r_n \to 0$, последовательность $\{x_n\}_{n=1}^\infty$ фундаментальна. Тогда, поскольку пространство $X$ полно, для некоторого $x^* \in X$ выполнено $x_n \to x^*$. Но каждый шар $\overline{B}(x_N,\, r_N)$ содержит все точки из последовательности $\{x_n\}_{n = 1}^\infty$, начиная с номера $N$, тогда, в силу его замкнутости, он также содержит точку $x^*$.

	Значит, $\{x^*\} \subset \bigcap_{n = 1}^\infty \overline{B}(x_n,\, r_n)$. Наконец, в силу условия $r_n \to +0$, других точек в пересечении быть не может.
\end{proof}

\begin{theorem}
	Теорема Бэра.

	Пусть $X$ -- полное метрическое пространство. Тогда $X$ нельзя представить в виде $X = \bigcup_{n = 1}^\infty M_n$, где множества $M_n \subset X$ -- не плотные ни в одном шаре в $X$ (нигде не плотные)
\end{theorem}

\begin{proof}
	Предположим противное, то есть $X$ имеет такой вид, как в условии. Положим $r_0 := 1$ и выберем произвольный шар $\overline{B}(x_0,\, r_0) \subset X$. Поскольку $M_1$ неплотно в $\overline{B}(x_0,\, r_0)$, то
	\[
		(X \setminus \overline{M}_1) \cap \overline{B}(x_0,\, r_0) \neq \emptyset
	\], поэтому можно выбрать шар
	\[
		\overline{B}(x_1,\, r_1) \subset \overline{B}(x_0,\, r_0) :\: \overline{B}(x_1,\, r_1) \cap \overline{M}_1 = \emptyset
	\]
	Можно считать, что $r_1 \leq \frac{1}{2}$. Повторим данное упражнение счётное количество раз\dots

	Рассмотрим полученную последовательность вложенных шаров $\{\overline{B}(x_n,\,r_n)\}_{n = 1}^\infty$. Поскольку $r_n \leq \frac{1}{2^n} \to 0$, то, по предыдущей теореме, для некоторой точки $x^* \in X$ выполнено равенство
	\[
		\{x^*\} = \cap_{n = 0}^\infty\overline{B}(x_n,\, r_n)
	\]
	По предположению, $X = \cup M_n$, поэтому $\exists n :\: x^* \in M_n$, но по построению
	\[
		\overline{B}(x^*,\, r_n) \cap \overline{M}_n = \emptyset
	\]
	противоречие.
\end{proof}

\subsection{Принцип сжимающих отображений.}
\begin{theorem}
	Теорема Банаха. Принцип сжимающих отображений.

	Пусть $X$ -- полное метрическое пространство, $f :\: X \to X$ -- отображение такое, что выполнено следующее условие:
	\[
		\exists \alpha \in (0,\, 1) \: \forall x,\, y \in X :\: \rho(f(x),\, f(y)) \leq \alpha\rho(x,\,y)
	\]
	Тогда
	\[
		\exists ! x^* :\: f(x^*) = x^*
	\]
\end{theorem}

\begin{proof}
	Существование. Зафиксируем $x_0 \in X$ и рассмотрим последовательность $\{x_n\}_{n = 1}^\infty$, где $x_{n + 1} = f(x_n)$. Поскольку для $k \in \mathbb{N}$ выполнено:
	\[
		\rho(x_{k + 1},\, x_k) = \rho(f(x_k),\, f(x_{k - 1})) \leq \alpha\rho(x_k,\, x_{k - 1}) = \alpha\rho(f(x_{k - 1},\, f(x_{k - 2}))) \leq \cdots \leq \alpha^k\rho(x_1,\, x_0)
	\]
	то по неравенству треугольника получаем
	\[
		\rho(x_{n + p},\, x_n) \leq \rho(x_{n + p},\, x_{n + p - 1}) + \cdots + \rho(x_{n + 1},\, x_n) \leq (\alpha^{n + p - 1} + \cdots + \alpha^n) \rho(x_1,\, x_0) \leq \frac{\alpha^n}{1 - \alpha}\rho(x_1,\, x_0)
	\]
	Так как $\alpha^n \overset{n \to +\infty}{\to} 0$, то $\{x_n\}_{n = 1}^\infty$ фундаментальна. Значит, из полноты пространства,
	\[
		\exists \lim_{n \to +\infty} x_n = x^*
	\]
	Переходя к пределу в равенстве $x_{n + 1} = f(x_n)$ и пользуясь непрерывностью $f$, получаем $f(x^*) = x^*$.

	Единственность. Предположим, что
	\[
		\exists y^* \neq x^* :\: f(y^*) = y^* \Rightarrow \rho(x^*,\, y^*) = \rho(f(x^*),\, f(y^*)) \overset{f \text{ сжим}}{\leq} \alpha\rho(x^*,\, y^*)
	\]
	Это возможно лишь когда $\rho(x^*,\, y^*) = 0 \Rightarrow x^* = y^*$.
\end{proof}

\section{Компактные метрические пространства}
\subsection{Компактность и центрированные системы замкнутых множеств}
\begin{definition}
	Метрическое пространство $X$ называется компактным, если
	\[
		\forall \{G_\alpha\}_{\alpha \in \mathcal{A}} \subset 2^X,\, G_\alpha \text{ - открытые} :\: \bigcup_{\alpha \in \mathcal{A}} G_\alpha = X :\: \exists \{\alpha_i\}_{i = 1}^n \subset \mathcal{A} :\: \bigcup_{i = 1}^n G_{\alpha_i} = X
	\]
\end{definition}

\begin{definition}
	Пусть $X$ -- метрическое пространство. Система $\{B_\alpha\}_{\alpha \in \mathcal{A}} \subset 2^X$ называется центрированной, если
	\[
		\forall \{\alpha_i\}_{i=1}^n \subset \mathcal{A} :\: \bigcap_{i = 1}^n B_{\alpha_i} \neq \emptyset
	\]
\end{definition}

\begin{theorem}
	Метрическое пространство $X$ компактно $\Leftrightarrow$ любая центрированная система замкнутых множеств в $X$ имеет непустое пересечение.
\end{theorem}

\begin{proof}
	Каждой системе открытых множеств $\{G_\alpha\}_{\alpha \in \mathcal{A}} \subset 2^X$ можно поставить в соответствие систему замкнутых множеств $\{F_\alpha\}_{\alpha \in \mathcal{A}} := \{X \setminus G_\alpha\}_{\alpha \in \mathcal{A}}$ и наоборот.

	Тогда любая система открытых множеств $\{G_\alpha\}_{\alpha \in \mathcal{A}}$, не содержащая конечного подпокрытия, не является покрытием $\Leftrightarrow$ любая центрированная система замкнутых множеств $\{F_\alpha\}_{\alpha \in \mathcal{A}}$ имеет непустое пересечение (накиньте на одну из частей утверждения дополнения и поймите, что это одно и то же).
\end{proof}

\subsection{Критерий компактности}
\begin{definition}
	Пусть $M$ -- некоторое множество в метрические пространстве $R$. Тогда множества $A$ из $R$ называется $\varepsilon$-сетью для $M$, если
	\[
		\forall x \in M \: \exists a \in A :\: \rho(x,\, a) \leq \varepsilon
	\]
\end{definition}

\begin{definition}
	Множество $M$ в метрическом пространстве $R$ называется ограниченным, если
	\[
		\exists B(x_0,\, \varepsilon) \supset M
	\]
\end{definition}

\begin{definition}
	Множество $M$ в метрическом пространстве $R$ называется вполне ограниченным, если для него при любом $\varepsilon > 0$ существует конечная $\varepsilon$-сеть.
\end{definition}

\begin{lemma}
	Из вполне ограниченности следует ограниченность.
\end{lemma}

\begin{proof}
	Из вполне ограниченности ограниченность получается, как объединение конечного числа ограниченных множеств.
\end{proof}

\begin{theorem}
	Критерий компактности.

	Пусть $X$ -- метрическое пространство. Тогда следующие условия эквивалентны:
	\begin{enumerate}
		\item $X$ компактно.
		\item $X$ полно и вполне ограниченно.
		\item Из любой последовательности $\{x_n\}_{n = 1}^\infty \subset X$ можно выделить сходящуюся подпоследовательность $\{x_{n_k}\}_{k = 1}^\infty$, ещё говорят, что $X$ -- секвенциально компактно.
		\item Любое бесконечное множество $M \subset X$ имеет предельную точку.
	\end{enumerate}
\end{theorem}

\begin{proof}
	\begin{itemize}
		\item ($1 \Rightarrow 2$) $X$ вполне ограниченно, поскольку для любого $\varepsilon > 0$ из открытого покрытия $\{B(x,\, \varepsilon)\}_{x \in X}$ по определению можно выделить конечное подпокрытие. Центры шаров этого подпокрытия и будут давать требуемую $\varepsilon$-сеть.

		      Пусть последовательность $\{x_n\}_{n=1}^\infty \subset X$ фундаментальна. Для каждого $n \in \mathbb{N}$ положим $A_n := \{x_n,\, x_{n + 1},\, \cdots\}$, тогда система $\{\overline{A}_n\}_{n=1}^\infty$ является центрированной системой замкнутых множеств. Система центрирована, потому что у любого конечного набора пересечением будет являться хвост, начинающийся с максимального из взятых индексов.

		      Поэтому можно выбрать точку $x_0 \in \cap_{n \in \mathbb{N}_+} \overline{A}_n$, причём $x_0 \in X$ по рассмотренному выше критерию компактности. В силу фундаментальности
		      \[
			      \forall \varepsilon > 0 \: \exists N \in \mathbb{N} \: \forall n > N :\: \overline{A}_n \subset \overline{B}(x_N,\, \varepsilon)
		      \]
		      откуда и $\rho(x_n,\, x_0) < \varepsilon \Rightarrow x_n \to_X x_0$.
		\item ($2 \Rightarrow 3$) Зафиксируем произвольную последовательность $\{x_n\}_{n = 1}^\infty \subset X$. Поскольку $X$ вполне ограниченно, то
		      \[
			      \forall \varepsilon > 0 \: \exists y \in X :\: \vert\{x_n\}_{n = 1}^\infty \cap B(y,\, \varepsilon)\vert = +\infty
		      \]
					По сути просто принцип Дирихле -- имеем конечное количество элементов $\varepsilon$-сети и бесконечное количество членов последовательности. Значит в один из шаров попадёт бесконечное число членов.

		      Будем применять это рассуждение сначала для всего пространства $X$, потом для шаров в $X$, содержащих бесконечно много точек из $\{x_n\}_{n = 1}^\infty$:
		      \begin{itemize}
			      \item Для $\varepsilon := 1$ выберем $\{x_k^1\}_{k=1}^\infty \subset \{x_n\}_{n=1}^\infty$ так, что $\{x_k^1\}_{k=1}^\infty \subset B(y_1,\, 1)$
			      \item Для $\varepsilon := \frac{1}{2}$ выберем $\{x_k^2\}_{k=1}^\infty \subset \{x_n^1\}_{n=1}^\infty \subset B(y_1,\, 1)$ так, что $\{x_k^2\}_{k=1}^\infty \subset B(y_2,\, \frac{1}{2})$
			      \item \dots
		      \end{itemize}
		      Рассмотрим диагональную последовательность $\{x_k^k\}_{k=1}^\infty \subset \{x_n\}_{n=1}^\infty$. По построению, она является фундаментальной, и в силу полноты пространства $X$, она сходится.
		\item ($3 \Rightarrow 1$) Проверим сначала, что $X$ вполне ограниченно. Предположим противное, то есть
		      \[
			      \exists \varepsilon_0 > 0 \: \forall \{\overline{B}(y_n,\, \varepsilon_0)\}_{n = 1}^N :\: \bigcup_{n = 1}^N \overline{B}(y_n,\, \varepsilon_0) \not\supset X
		      \]
		      Тогда можно выбрать точку $x_1 \in X$, затем точку $x_2 \in (X \setminus B(x_1,\, \varepsilon_0))$. По предположению, остаток, из которого берём элементы последовательности, никогда не будет пуст, поэтому получим последовательности с попарными расстояниями между точками не меньше $\varepsilon_0$, из которой, очевидно, нельзя выделить сходящуюся подпоследовательность -- противоречие.

		      Теперь проверим, что $X$ компактно. Предположим противное, то есть
		      \[
			      \exists \{G_\alpha\}_{\alpha \in \mathcal{A}},\, G_\alpha \text{ - открытое} \: \forall \{G_{\alpha_i}\}_{i = 1}^N :\: \bigcup_{i = 1}^N  G_{\alpha_i} \not\supset X
		      \]
		      Значит,
		      \[
			      \forall \varepsilon > 0 \: \exists x \in X  \: \forall \{G_{\alpha_i}\}_{i = 1}^N :\: \bigcup_{i = 1}^N  G_{\alpha_i} \not\supset B (x,\, \varepsilon)
		      \]
		      (если такого шара нет, то из вполне ограниченности, складывая конечные покрытия конечного числа шаров, получим конечное покрытие всего множества).

		      Выбирая такую точку $x_n$ для $\varepsilon := \frac{1}{n}$ при каждом $n \in \mathbb{N}$, получим последовательность $\{x_n\}_{n = 1}^\infty$, из которой можно выделить сходящуюся подпоследовательность $\{x_{n_k}\}_{k = 1}^\infty$. Пусть $x_{n_k} \to_X x_0 \in X$.
		      Тогда существует $\alpha_0 \in \mathcal{A}$ такое, что $x_0 \in G_{\alpha_0}$. Но множество $G_{\alpha_0}$ открыто, поэтому оно покрывает некоторую окрестность точки $x_0$, а значит и все шары $B(x_{n_k},\, \frac{1}{n_k})$, начиная с некоторого номера -- противоречие.
		\item ($3 \Rightarrow 4$) Зафиксируем бесконечное множество $M \subset X$, тогда, выбирая произвольным образом последовательность $\{x_n\}_{n = 1}^\infty \subset M$ и выделяя из неё сходящуюся подпоследовательность, получим требуемое.
		\item ($4 \Rightarrow 3$) Зафиксируем последовательность $\{x_n\}_{n = 1}^\infty$. Если множество её значений конечно, то в ней можно выделить стационарную подпоследовательность. Если же множество её значений бесконечно, то оно имеет предельную точку $x_0 \in X$, поэтому можно выбрать подпоследовательность $\{x_{n_k}\}$ такую, что $x_{n_k} \to_X x_0$
	\end{itemize}
\end{proof}

\subsection{Теорема Арцела-Асколи}
\begin{definition}
	Обозначим за $C(X,\, Y)$ множество непрерывных функций $f:\: X \to Y$.
\end{definition}

\begin{theorem}
	Теорема Кантора.

	Пусть $X$ -- компактное метрическое пространство, и функция $f \in C(X,\, \mathbb{R})$. Тогда $f$ равномерно непрерывна на $X$.
\end{theorem}

\begin{proof}
	Предположим противное, то есть выполнено следующее:
	\[
		\exists \varepsilon_0 > 0 \: \forall \delta > 0 \: \exists x,\,y \in X,\, \rho(x,\, y) < \delta :\: \vert f(x) - f(y)\vert \geq \varepsilon_0
	\]
	Выбирая $\delta := \frac{1}{n}$ для каждого $n \in \mathbb{N}$, получим последовательности $\{x_n\}_{n=1}^\infty,\, \{y_n\}_{n=1}^\infty$. Поскольку $X$ компактно, можно выделить из них сходящиеся подпоследовательности $\{x_{n_k}\}_{k=1}^\infty$, $\{y_{n_k}\}_{k=1}^\infty$, причём сходятся они к одной и той же точке $x_0 \in X$ по построению. Однако для любого $k \in \mathbb{N}$ выполнено $\vert f(x_{n_k}) - f(y_{n_k})\vert \geq \varepsilon_0$ -- противоречие.
\end{proof}

\begin{theorem}
	Арцела-Асколи.

	Пусть $X$ -- компактное метрическое пространство, $M \subset C(X,\, \mathbb{R})$. Тогда множество $M$ вполне ограниченно $\Leftrightarrow$ множество $M$ ограниченно и выполнено условие равностепенной непрерывности:
	\[
		\forall \varepsilon > 0 :\: \exists \delta > 0 :\: \forall x,\, y \in X,\, \rho(x,\, y) < \delta :\: \forall f \in M :\: \vert f(x) - f(y)\vert < \varepsilon
	\]
\end{theorem}

\begin{proof}
	($\Rightarrow$) Мы уже доказывали, что из вполне ограниченности следует обычная ограниченность, проверим условие равностепенной непрерывности. Зафиксируем произвольное $\varepsilon > 0$ и выберем конечным набор функций $\phi_1,\,\cdots,\,\phi_n \in C(X,\, \mathbb{R})$, образующий $\varepsilon$-сеть.

	По теореме Кантора, каждая из этих функций равномерно непрерывна. Пусть $\delta_1,\,\cdots,\,\delta_n > 0$ -- числа, соответствующие выбранному $\varepsilon$ в определении равномерной непрерывности:
	\[
		\forall \varepsilon > 0 \: \exists \delta_k > 0 \: \forall x,\,y \: \rho(x,\, y) < \delta_k :\: \vert\phi_k(x) -\phi_k(y)\vert < \varepsilon
	\]
	Тогда для $\delta := \min\{\delta_1,\,\cdots,\, \delta_n\}$ выполнено требуемое:
	\[
		\vert f(x) - f(y)\vert \leq \vert f(x) - \phi_k(x)\vert + \vert \phi_k(x) - \phi_k(y)\vert + \vert \phi_k(y) - f(y)\vert < 3\varepsilon
	\]
	($\Leftarrow$) Поскольку множество $M$ ограниченно, то существует $C > 0$ такое, что
	\[
		\forall f \in M :\: \|f\| = \sup_{x \in [a,\,b]} \vert f(x)\vert \leq C
	\]
	Зафиксируем произвольное $\varepsilon > 0$ и выберем по нему $\delta > 0$ из условия равностепенной непрерывности.

	Разобьём отрезок $[a,\,b]$ на части длины меньше $\delta$ точками
	\[
		a = x_0 < x_1 < \cdots < x_n = b
	\]
	, а отрезок $[-C,\, C]$ -- на части длины меньше $\varepsilon$ точками
	\[
		-C = y_0 < y_1 < \cdots < y_m = C
	\]
	и рассмотрим конечное множество $L$ кусочно линейных функций, построенных по всевозможным наборам точек вида
	\[
		\{(x_j,\, y_{i_k})\}_{j = 0}^n,\, i_k \in \overline{0,\,m}
	\]
	Из такого построению становится очевидно, что
	\[
		\forall f \in M \: \exists \phi \in L \: \forall i \in \{0,\,\cdots,\,n\} :\: \vert f(x_i) - \phi(x_i)\vert < \varepsilon
	\]
	Рассмотрим произвольную точку $x \in [a,\,b]$ и выберем $i \in \{0,\,\cdots,\, n-1\}$ такое, что $x \in [x_i,\, x_{i + 1}]$, тогда:
	\[
		\vert f(x) - \phi(x)\vert \leq \vert f(x) - f(x_i)\vert + \vert f(x_i) - \phi(x_i)\vert + \vert \phi(x) - \phi(x_i)\vert < 2\varepsilon + \vert \phi(x_{i + 1}) - \phi(x_i)\vert
	\]
	Первое слагаемое меньше $\varepsilon$ из равностепенной непрерывности, а второе по построению $\phi$. Оценим слагаемое $\vert \phi(x_{i + 1}) - \phi(x_i)\vert$ отдельно:
	\[
		\vert \phi(x_{i + 1}) - \phi(x_i)\vert \leq \vert f(x_{i + 1}) - \phi(x_{i + 1})\vert + \vert f(x_{i + 1}) - f(x_i)\vert + \vert f(x_i) - \phi(x_i)\vert < 3\varepsilon
	\]
	Таким образом, $\sup_{x \in [a,\,b]}\vert f(x) - \phi(x)\vert < 5\varepsilon$. Значит, построенное множество $L$ образует конечную $5\varepsilon$-сеть для множества $M$, тогда, в силу произвольности выбора числа $\varepsilon$, множество $M$ вполне ограниченно.
\end{proof}

\section{Линейные нормированные пространства}
\subsection{Теорема Рисса}
\begin{definition}
	Линейным нормированным пространством над полем $\mathbb{K}$, где $\mathbb{K} = \mathbb{R}$ или $\mathbb{K} = \mathbb{C}$, называется линейное пространство $E$ над $\mathbb{K}$ с функцией $\|\cdot\| :\: E \to \mathbb{R}$, обладающей следующими свойствами:
	\begin{enumerate}
		\item $\forall x \in E :\: \|x\| \geq 0$, причём $\|x\| = 0 \Leftrightarrow x = 0$
		\item $\forall x \in E \: \forall \alpha \in \mathbb{K} :\: \|\alpha x\| = \vert\alpha\vert\|x\|$
		\item $\forall x,\,y \in E :\: \|x + y\| \leq \|x\| + \|y\|$ (неравенство треугольника)
	\end{enumerate}
	Функция $\|\cdot\|$ называется нормой на пространстве $E$.
\end{definition}

\begin{lemma}\label{NormCont}
	Норма непрерывна, как функция $E \to \mathbb{R}_+$
\end{lemma}

\begin{proof}
	\[
		\forall \{x_n\}_{n=1}^\infty :\: x_n \overset{n \to +\infty}{\to} x \Rightarrow \rho(x_n,\, x) = \|x_n - x\| \to 0
	\]
	Дважды воспользуемся неравенством треугольника:
	\begin{itemize}
		\item $\|x_n\| \leq \|x_n - x\| + \|x\|$
		\item $\|x\| \leq \|x_n - x\| + \|x_n\|$
	\end{itemize}
	Таким образом, $\vert\|x_n\| - \|x\|\vert \to 0$, то есть $\|x_n\| \to \|x\|$, что и требовалось.
\end{proof}

\begin{definition}
	Пусть $E$ -- линейное нормированное пространство. Множество $L \subset E$ называется:
	\begin{itemize}
		\item Линейным многообразием в $E$, если $L$ замкнуто относительно сложения и умножения на скаляры из $\mathbb{K}$.
		\item Подпространством в $E$, если $L$ является линейным многообразием в $E$ и при этом замкнуто.
	\end{itemize}
\end{definition}

\begin{lemma}
	О почти перпендикуляре.

	Пусть $E$ -- линейное нормированное пространство, $L \subsetneq E$ -- подпространство. Тогда
	\[
		\forall \varepsilon > 0 \: \exists y \in E \: \|y\| = 1 :\: \rho(y,\, L) := \inf_{z \in L} \|y - z\| \geq 1 - \varepsilon
	\]
\end{lemma}

\begin{proof}
	Зафиксируем $y_0 \in E \setminus L$ и положим $d := \rho(y_0,\, L) > 0$. Выберем вектор $z_0 \in L$ такой, что $d \leq \|y_0 - z_0\| \leq d(1 + \varepsilon)$ и покажем, что подходит вектор $y := \frac{y_0 - z_0}{\|y_0 - z_0\|} := \alpha(y_0 - z_0)$. Действительно, $\|y\| = 1$, и для любого $z \in L$ выполнены неравенства:
	\[
		\|y - z\| = \|\alpha(y_0 - z_0) - z\| = \vert\alpha\vert \|y_0 - \left(z_0 + \frac{1}{\alpha}z\right)\| \geq \vert\alpha\vert d \geq \frac{d}{d(1 + \varepsilon)} \geq 1 - \varepsilon
	\]
	Для осмысления последних переходов вникните в следующее утверждение:
	\[
		\|y_0 - z_0\| \leq d(1 + \varepsilon) \Rightarrow \alpha = \frac{1}{\|y_0 - z_0\|} \geq \frac{1}{d(1 + \varepsilon)}
	\]
	А также не забывайте про разложение Тейлора:
	\[
		\frac{1}{1 + x} \geq 1 - x
	\]
\end{proof}

\begin{definition}
	Пусть $E$ -- линейное нормированное пространство, $M \subset E$. Линейной оболочкой множества $M$ называется множество следующего вида:
	\[
		\langle M\rangle := \left\{\sum_{k = 1}^n\alpha_im_i \:\vert\: \{\alpha_i\}_{i = 1}^n \subset \mathbb{K},\, \{m_i\}_{i = 1}^n \subset M\right\}
	\]
\end{definition}

\begin{theorem}
	Рисса.

	Пусть $E$ -- линейное нормированное пространство. Тогда единичная сфера $S(0,\,1)$ компактна в $E \Leftrightarrow \dim E < +\infty$
\end{theorem}

\begin{proof}
	($\Leftarrow$) По эквивалентности норм конечномерных пространствах (теорема будет далее), все нормы на конечномерном линейном пространстве эквивалентны, а относительно евклидовой нормы сфера $S(0,\,1)$ компактна.

	($\Rightarrow$) От противного. Зафисиксируем $\varepsilon > 0$ и построим последовательность $\{x_n\}_{n=1}^\infty \subset S(0,\,1)$ с попарными расстояниями не меньше $1 - \varepsilon$, из чего будет следовать, что сфера $S(0,\,1)$ не вполне ограничена и потому не компактна:
	\begin{itemize}
		\item Выберем $x_1 \in S(0,\,1)$ произвольным образом
		\item По предыдущей лемме выберем $x_2 \in S(0,\,1) \setminus \langle x_1\rangle$ такое, что $\rho(x_2,\, \langle x_1\rangle) \geq 1 - \varepsilon$, причём утверждение применимо, так как линейная оболочка конечномерна, а линейная оболочка конечного числа элементов к тому же замкнута, а значит это действительно подпространство.
		\item \dots
	\end{itemize}
	Поскольку $\dim E = +\infty$, то процесс не закончится, и будет получена искомая последовательность $\{x_n\}_{n = 1}^\infty$.
\end{proof}

\subsection{Характеристическое свойство евклидовых пространств\dots}
\begin{definition}
	Евклидовым пространством над полем $\mathbb{K}$, где $\mathbb{K} = \mathbb{R}$ или $\mathbb{K} = \mathbb{C}$, называется линейное пространство $E$ над $\mathbb{K}$ с функцией $(\cdot,\, \cdot) :\: E^2 \to \mathbb{K}$, обладающее следующими свойствами:
	\begin{enumerate}
		\item $\forall x \in E :\: (x,\, x) \geq 0$, причём $(x,\, x) = 0 \Leftrightarrow x = 0$
		\item $\forall x,\,y \in E :\: (x,\,y) = \overline{(y,\, x)}$
		\item $\forall x,\,y,\, z \in E :\: \forall \alpha,\, \beta \in \mathbb{K} :\: (\alpha x + \beta y,\, z) = \alpha(x,\,z) + \beta(y,\,z)$
	\end{enumerate}
\end{definition}

\begin{definition}
	Полное линейное нормированное пространство $E$ называется банаховым.
\end{definition}

\begin{definition}
	Полное евклидово пространство $H$ называется гильбертовым.
\end{definition}

\begin{theorem}
	Характеристическое свойство Евклидовых пространств.

	Пусть $E$ -- линейное нормированное пространство. Тогда норма в $E$ порождается скалярным произведением $\Leftrightarrow$
	\[
		\forall x,\, y \in E :\: \|x + y\|^2 + \|x - y\|^2 = 2\|x\|^2 + 2\|y\|^2
	\]
\end{theorem}

\begin{proof}
	($\Rightarrow$) Распишем квадраты норм, как скалярные произведения элементов самих на себя:
	\begin{align*}
		\langle x + y,\, x + y\rangle + \langle x -y ,\, x - y\rangle = \langle x,\, x\rangle + 2\langle x,\, y\rangle + \langle y,\, y\rangle + \langle x,\, x\rangle - 2\langle x,\, y\rangle + \langle y,\, y\rangle = \\
		2\|x\|^2 + 2\|y\|^2
	\end{align*}

	($\Leftarrow$) Без доказательства, очень-очень сложно
\end{proof}

\subsection{Эквивалентность норм в конечномерном пространстве\dots}
\begin{definition}
	Две нормы $p,\, q :\: V \to \mathbb{R}_+$ над пространством $V$ называются эквивалентными, если
	\[
		\exists C_1,\, C_2 > 0 \: \forall x \in V :\: C_1p(x) \leq q(x) \leq C_2p(x)
	\]
\end{definition}

\begin{lemma}
	Пусть $E$ -- линейное нормированное пространство, и $\dim E < +\infty$. Тогда любые две нормы на $E$ эквивалентны.
\end{lemma}

\begin{proof}
	(Доказательство для $\mathbb{K} = \mathbb{R}$)

	Поскольку $E$ конечномерно, то в нём можно выбрать максимальную по включению линейно независимую систему $e = \{e_1,\,\cdots,\,e_n\} \subset E$, тогда $e$ будет являться базисом в $E$. Для произвольного элемента $x \in E$, имеющего в базисе $e$ координатный столбец $\alpha \in \mathbb{R}^n$, зададим его евклидову норму следующим образом:
	\[
		\|x\|_e = \sqrt{\sum_{k = 1}^n \alpha_k^2}
	\]
	Зафиксируем произвольную норму $\|\cdot\|$ и докажем, что она эквивалентна норме $\|\cdot\|_e$:
	\begin{enumerate}
		\item Покажем, что $\|\cdot\| < C\|\cdot\|_e$ для некоторого $C > 0$. Для произвольного $x \in E$, имеющего в базисе $e$ координатный столбец $\alpha \in \mathbb{R}^n$, выполнено следующее:
		      \[
			      \|x\| = \left\|\sum_{k = 1}^n \alpha_ke_k\right\| \leq \max_{1 \leq k \leq n}\|e_k\| \left(\sum_{k = 1}^n \vert \alpha_k\vert\right)
		      \]
		      Поскольку для любого $k \in \{1,\, \cdots,\,n\}$ выполнено $\alpha_k < \|x\|_e$, то достаточно взять число $C := n\cdot\max_{1\leq k\leq n}\|e_k\|$
		\item Покажем теперь, что $\|\cdot\|_e < \tilde{C}\|\cdot\|$ для некоторого $\tilde{C} > 0$. Предположим противное, тогда
		      \[
			      \forall n \in \mathbb{N} \: \exists x_n \in E :\: \|x_n\|_e > n\|x_n\|
		      \]
		      Можно без ограничения общности считать, что $\|x_n\|_e = 1$ для любого $n \in \mathbb{N} \Rightarrow \|x_n\| < \frac{1}{n}$

		      Поскольку последовательность $\{x_n\}_{n = 1}^\infty$ содержится в единичной сфере $S_e(0,\,1)$ и относительно евклидовой нормы сфера компактна, можно выделить из $\{x_n\}_{n = 1}^\infty$ подпоследовательность $\{x_{n_k}\}_{k = 1}^\infty$, сходящуюся относительно евклидовой нормы. Тогда
		      \[
			      \exists x \in S_e(0,\,1) :\: \|x_{n_k} - x\|_e \overset{k \to +\infty}{\to} 0
		      \]
		      Тогда в силу предыдущего пункта $\|x_{n_k} - x\| \to 0$. Но по построению $\|x_{n_k}\| < \frac{1}{n_k} \overset{k \to +\infty}{\to} 0$, поэтому $x = 0$ -- противоречие с тем, что $x \in S_e(0,\,1)$.
	\end{enumerate}
\end{proof}

\begin{corollary}
	Пусть $E$ -- линейное нормированное пространство $x_1,\,\cdots,\,x_n \in E$. Тогда линейная оболочка $L := \langle x_1,\,\cdots,\, x_n\rangle$ образует подпространство в $E$.
\end{corollary}

\begin{proof}
	Заметим, что $\dim L < +\infty$, и по предыдущему утверждению сужение нормы из $E$ на $L$ эквивалентно евклидовой норме. Относительно евклидовой нормы конечномерное пространство полно, поэтому и $L$ полно относительно нормы из $E$. Следовательно, $L$ замкнуто, как подмножество в $E$.
\end{proof}

\begin{definition}
	Линейным топологическим пространством называется топологическое пространство $X$ с определёнными на нём операциями сложения и умножения на числа из поля $\mathbb{K}$, непрерывными на $X$.
\end{definition}

\begin{example}
	Любое линейное нормированное пространство $E$ является линейным топологическим.
\end{example}

\subsection{Теорема Рисса о проекции}
\begin{definition}
	Пусть $E$ -- линейное нормированное пространство, $L \subset E$ -- линейное нормированное подпространство, $h \in E$. Элементом наилучшего приближения для $h$ называется $x \in L$ такой, что $\|h - x\| = \rho(h,\, L) = \inf_{y \in L}\|h -y\|$
\end{definition}

\begin{lemma}
	Пусть $H$ -- гильбертово пространство, $M \subset H$ -- подпространство в $H$. Тогда для любого $h \in H$ существует единственный элемент наилучшего приближения $x \in M$.
\end{lemma}

\begin{proof}
	Зафиксируем $h \in H$. Сначала докажем, что элемент наилучшего приближения для $h$ существует.

	Положим $d := \rho(h,\, M)$. Если $d = 0$, то, в силу замкнутости множества $M$, выполнено $h \in M$, и в качестве элемента наилучшего приближения для $h$ подходит сам $h$. Иначе выберем $\{x_n\}_{n = 1}^\infty \subset M$ такую, что $\|h - x_n\| \overset{n \to +\infty}{\to} d$. По равенству параллелограмма,
	\[
		\forall n,\,m \in \mathbb{N} :\: \|x_n - x_m\|^2 = 2 \|h - x_n\|^2 + 2\|h - x_m\|^2 - 4\left\|h - \frac{x_n + x_m}{2}\right\|^2
	\]
	Поскольку $\|h - x_n\|^2 \overset{n \to +\infty}{\to} d^2,\, \|h - x_m\|^2 \overset{m \to +\infty}{\to} d^2$ и выполнено неравенство $\|h - \frac{x_n + x_m}{2}\|^2 \geq d^2$ (норма любой разности $h$ и чего-либо из $M$ будет не меньше расстояния) , последовательность $\{x_n\}_{n = 1}^\infty$ фундаментальна. Поскольку пространство $H$ полно, а $M$ замкнуто, то существует $x \in M$ такой, что $x_n \to_H x$, причём, в силу непрерывности нормы (\ref{NormCont}), $\|h - x\| = d$

	Покажем теперь, что элемент наилучшего приближения для $h$ единственен. Пусть для некоторого $y \in M$ тоже выполнено равенство $\|h - y\| = d$, тогда, по равенству параллелограмма, выполнено следующее:
	\[
		4d^2= 2\|h - x\|^2 + 2\|h - y\|^2 = \|x - y\|^2 + 4\left\|h - \frac{x + y}{2}\right\|^2 \geq \|x - y\|^2 + 4d^2
	\]
	Для данной цепочки мы воспользовались следующими равенствами:
	\[
		h - x + h - y = 2h - x - y = 2\left(h - \frac{x + y}{2}\right);\; h - x - h + y = -(x - y)
	\]

	Итак, $\|x - y\| =0 \Rightarrow x = y$, что и требовалось.
\end{proof}

\begin{definition}
	Пусть $E$ -- евклидово пространство, $S \subset E$. Аннулятором множества $S$ называется следующее множество:
	\[
		S^\bot := \{y \in E \:\vert\: \forall x \in S :\: (x,\,y) = 0 \}
	\]
\end{definition}

\begin{note}
	Легко проверить, что $S^\bot$ является подпространством в $E$. Кроме того, выполнены равенства
	\[
		S^\bot = \langle S\rangle^\bot = \overline{\langle S\rangle}^\bot
	\]
\end{note}

\begin{theorem}
	Рисса, о проекции.

	Пусть $H$ -- гильбертово пространство, $M \subset H$ -- подпространство в $H$. Тогда $H = M \oplus M^\bot$.
\end{theorem}

\begin{proof}
	Покажем сначала, что $H = M + M^\bot$ (разложение существует, но м.б. не единственно).

	Зафиксируем $h \in H$, и по утверждению о наилучшем приближении, выберем $x \in M$. Положим $y := h - x$ и $d := \|y\|$, тогда для произвольного $m \in M$ и произвольного $\alpha \in \mathbb{R}\setminus \{0\}$ выполнено следующее:
	\[
		d^2 = \|h - x\|^2 \leq \|h - (x + \alpha m)\|^2 = d^2 - 2\alpha(h - x,\, m) + \alpha^2\|m\|^2 \Rightarrow (y,\, m) \leq \frac{\alpha}{2}\|m\|^2
	\]
	В силу произвольности $\alpha$ получаем, что $(y,\, m) = 0 \Rightarrow y \in M^\bot$, причём $h = x+ y$. Значит, выполнено равенство $H = M + M^\bot$.

	Проверим теперь, что рассматриваемая сумма действительно прямая. Если $z \in M \cap M^\bot \Rightarrow (z,\,z) = 0 \Rightarrow z = 0$, что и означает требуемое.
\end{proof}

\subsection{Сепарабельные гильбертовы пространства}
\begin{definition}
	Пусть $E$ -- линейное нормированное пространство. Система $\{e_n\}_{n=1}^\infty \subset E$ называется базисом Шаудера в $E$, если
	\[
		\forall x \in E \: \exists! \{\alpha_n\}_{n = 1}^\infty \subset \mathbb{K} :\: x = \sum_{n = 1}^\infty \alpha_ne_n
	\]
\end{definition}

\begin{lemma}
	Неравенство Бесселя.

	Пусть $E$ -- евклидово пространство, элементы $\{e_i\}_{i=1}^\infty \subset E$ образуют ортонормированную систему. Тогда для любого $x \in E$ выполнено следующее неравенство:
	\[
		\sum_{k = 1}^\infty \vert(x,\, e_k)\vert^2 \leq \|x\|^2
	\]
\end{lemma}

\begin{proof}
	Достаточно заметить, что в силу ортонормированности системы $\{e_k\}_{k = 1}^\infty$ выполнено следующее равенство:
	\[
		0 \leq \left\|x - \sum_{k = 1}^n (x,\, e_k)e_k\right\|^2 = \|x\|^2 - \sum_{k = 1}^n \vert(x,\, e_k)\vert^2
	\]
	Поскольку левая часть равенства неотрицательна, то неотрицательна и правая часть, и верно это даже после предельного перехода $n \to +\infty$.
\end{proof}

\begin{lemma}
	Пусть $E$ -- евклидово пространство, элементы $e_1,\,\cdots,\, e_k \in E$ образуют ортонормированную систему. Тогда
	\[
		\forall x \in E \: \forall \{\alpha_k\}_{k = 1}^n \subset \mathbb{K} :\: \left\|x - \sum_{k =1}^n\alpha_ke_k\right\| \geq \left\|x - \sum_{k = 1}^n(x,\,e_k)e_k\right\|
	\]
	Более того, равенство в неравенстве выше достигается тогда и только тогда, когда

	\[
		\forall k \in \overline{1,\, n} :\: \alpha_k = (x,\, e_k)
	\]
\end{lemma}

\begin{proof}
	В силу ортонормированности системы $\{e_k\}_{k = 1}^n$, выполнены следующие равенства:
	\begin{align*}
		\left\|x - \sum_{k=1}^n \alpha_ke_k\right\| = \left(x - \sum_{k = 1}^n\alpha_ke_k,\, x - \sum_{k = 1}^n \alpha_ke_k\right) = \\
		= \|x\|^2 - 2\sum_{k = 1}^n \alpha_k(x,\,e_k) + \sum_{k = 1}^n \alpha_k^2 = \left\|x - \sum_{k = 1}^n (x,\, e_k)e_k\right\|^2 + \sum_{k = 1}^n ((x,\, e_k) - \alpha_k)^2
	\end{align*}
\end{proof}

\begin{theorem}
	Пусть $H$ -- сепарабельное гильбертово пространство, $\dim H = +\infty$ и $e = \{e_n\}_{n=1}^\infty$ -- ортонормированная система. Тогда следующие условия эквивалентны:
	\begin{enumerate}
		\item $e$ -- ортонормированный базис в $H$
		\item $\overline{\langle e\rangle} = H$, то есть $e$ -- полная система.
		\item Для любого $h \in H$ выполнено равенство Парсеваля:
		      \[
			      \|h\|^2 = \sum_{n = 1}^\infty \vert (h,\, e_n)\vert^2
		      \]
		\item $e^\bot = \{0\}$
	\end{enumerate}
\end{theorem}

\begin{proof}
	\begin{itemize}
		\item ($1 \Rightarrow 2$) Очевидно из определения базиса.
		\item ($2 \Rightarrow 1$) Зафиксируем произвольный $h \in H$ и произвольное $\varepsilon > 0$. По условию, существует конечный набор $\alpha_1,\,\cdots,\,\alpha_n$ такой, что $\|h - \sum_{k = 1}^n \alpha_ke_k\| < \varepsilon$. Тогда, по предыдущей лемме, выполнено также неравенство $\|h - \sum_{k = 1}^n (h,\,e_k)e_k\| < \varepsilon$. Тогда, в силу произвольности числа $\varepsilon$, выполнено равенство $h = \sum_{n = 1}^\infty (h,\, e_n)e_n$.

		      Проверим теперь, что разложение элемента $h$ единственно. Пусть для некоторого набора $\{\beta_n\}_{n=1}^\infty \subset \mathbb{K}$ выполнено равенство $h = \sum_{n = 1}^\infty \beta_ne_n$. Тогда для любого $k \in \mathbb{N}$, скалярно умножая частичную сумму ряда $\sum_{n = 1}^\infty \beta_ne_n$ на $e_k$ и переходя к пределу, получаем, что $\beta_k = (h,\,e_k)$, что и означает требуемое.
		\item ($1 \Leftrightarrow 3$) Уже было замечено, что для любого $h \in H$ и любого $k \in \mathbb{N}$ выполнено следующее:
		      \[
			      \left\|h - \sum_{k = 1}^n (h,\, e_k)e_k\right\| = \|h\|^2 - \sum_{k = 1}^n \vert(h,\, e_k)\vert^2
		      \]
		      Значит, $h = \sum_{n = 1}^\infty (h,\,e_n)e_n \Leftrightarrow \|h\|^2 = \sum_{n = 1}^\infty \vert (h,\, e_n)\vert^2$, и единственность разложения элемента $h$ дказывается так же, как и в импликации выше.

		\item ($2 \Leftrightarrow 4$) Из теоремы Рисса (о проекции) и равенства $e^\bot = \overline{\langle e\rangle}^\bot = H^\bot$ получаем требуемое.
	\end{itemize}
\end{proof}

\section{Линейные ограниченные операторы в линейных нормированных пространствах}
\subsection{Связь непрерывности и ограниченности линейного оператора}
\begin{definition}
	Пусть $E_1,\,E_2$ -- линейные нормированные пространства над полем $\mathbb{K}$ ($\mathbb{R}$ или $\mathbb{C}$). Тогда $A :\: E_1 \to E_2$ будем называть оператором, а $f :\: E \to \mathbb{K}$ -- функционалом.
\end{definition}

\begin{definition}
	Оператор $A$ называется ограниченным, если для любого ограниченного $M \subset E_1$ образ $A(M)$ ограничен в $E_2$.
\end{definition}

\begin{definition}
	Для линейных операторов можно ввести следующие определения:
	\begin{itemize}
		\item Образ оператора:
		      \[
			      \text{Im } A = \{y \in E_2 \:\vert\: \exists x \in E_1 :\: Ax = y\}
		      \]
		\item Ядро оператора:
		      \[
			      \ker A = \{x \in E_1 \:\vert\: Ax = 0\}
		      \]
	\end{itemize}
\end{definition}

\begin{definition}
	Линейный оператор $A$ называется непрерывным, если для любой последовательности $x_n \to x$ выполнено $Ax_n \to Ax$.
\end{definition}

\begin{theorem}
	Пусть $E_1,\,E_2$ -- линейные нормированные пространства. $A :\: E_1 \to E_2$ -- линейный оператор. Тогда $A$ -- ограниченный тогда и только тогда, когда $A$ -- непрерывный.
\end{theorem}

\begin{proof}
	($\Rightarrow$) Как мы знаем, $x_n \to x \Leftrightarrow \|x_n - x\| \to 0$. Тогда
	\[
		\|Ax_n - Ax\|_{E_2} = \|A(x_n - x)\|_{E_2} \leq \|A\| \cdot\|x_n - x\|_{E_1} \to 0
	\]
	($\Leftarrow$) Предположим противное, то есть $A$ не является ограниченным:
	\[
		\forall K \exists x :\: \|Ax\|_{E_2} > K\|x\|_{E_1}
	\]
	Пусть $K$ пробегает все натуральные числа, тогда образуется последовательность $\{x_n\}_{n=1}^\infty$ такая, что $\|Ax_n\|_{E_2} > n\|x_n\|_{E_1}$. Все $x_n$, очевидно, ненулевые. Рассмотрим последовательность $y_n = \frac{1}{n}\frac{x_n}{\|x_n\|_{E_1}} \to 0$.
	\[
		\forall n \in \mathbb{N} :\: \|Ay_n\|_{E_2} = \frac{\|Ax_n\|_{E_2}}{n\|x_n\|_{E_1}} > 1
	\]
	Но из-за непрерывности оператора $Ay_n \to A0 = 0$. Противоречие.
\end{proof}

\subsection{Топологии и сходимости в пространстве операторов\dots}
\begin{lemma}
	Если $A$ -- линейный оператор, то следующие условия эквивалентны:
	\begin{enumerate}
		\item $A$ -- ограниченный
		\item $\exists K :\: \|Ax\|_{E_2} \leq K\|x\|_{E_1}$
		\item Образ единичного шара под действием оператора $A$ ограничен
	\end{enumerate}
\end{lemma}

\begin{proof}
	($1 \Rightarrow 2$) Если $A$ -- ограниченный, то образ любого ограниченного множества ограничен. В частности, образ единичного шара $B$ ограничен. То есть
	\[
		\exists K \: \forall x \neq 0 :\: \left\|A\left(\frac{x}{\|x\|_{E_1}}\right)\right\|_{E_2} \leq K
	\]
	В силу линейности оператора:
	\[
		\exists K \: \forall x \neq 0 :\: \|Ax\|_{E_2} \leq K\|x\|_{E_1}
	\]
	($2 \Rightarrow 3$) Если
	\[
		\exists K :\: \|Ax\|_{E_2} \leq K\|x\|_{E_1}
	\]
	, то $\forall x :\: \|x\|_{E_1} \leq 1$ получаем, что $\|Ax\|_{E_2} \leq K$. То есть образ единичного шара ограничен.
	($3 \Rightarrow 1$)
	\[
		\exists K \: \forall x,\, \|x\|_{E_1} = 1 :\: \|Ax\|_{E_2} \leq K
	\]
	Пусть $M \subset E_1$ ограничено, то есть лежит в шаре радиуса $R$. Далее считаем, что $x \neq 0$:
	\[
		\forall x \in M :\: \left\|A\left(\frac{x}{\|x\|_{E_1}}\right)\right\| \leq K \Rightarrow \forall x \in M :\: \|Ax\|_{E_2} \leq K\|x\|_{E_1} \leq K\cdot R
	\]
\end{proof}

\begin{definition}
	Нормой линейного ограниченного оператора $A$ называется
	\[
		\|A\| = \inf\{K \:\vert\: \forall x :\: \|Ax\|_{E_2} \leq K\|x\|_{E_1}\}
	\]
\end{definition}

\begin{definition}
	$\mathcal{L}(E_1,\, E_2)$ -- пространство линейных ограниченных операторов, действующих из $E_1$ в $E_2$. Оно образует линейное пространство над $\mathbb{K}$.
\end{definition}

\begin{definition}
	Двойственное или сопряжённое пространство -- это
	\[
		E^* = \mathcal{L}(E,\, \mathbb{K})
	\]
	где $\mathbb{K} = \mathbb{R}$ или $\mathbb{C}$, $E$ -- линейное нормированное пространство.
\end{definition}

\begin{theorem}
	Пусть $E_1,\, E_2$ -- линейные нормированные пространства. Тогда
	\begin{enumerate}
		\item $\mathcal{L}(E_1,\,E_2)$ -- линейное нормированное пространство с нормой $\|A\|$.
		\item Если $E_2$ -- банахово, то $\mathcal{L}(E_1,\, E_2)$ -- банахово.
	\end{enumerate}
\end{theorem}

\begin{proof}
	\begin{enumerate}
		\item Линейность данного пространства очевидна. Проверим неравенство треугольника для нормы:
		      \[
			      \|A_1 + A_2\| = \sup_{\|x\| = 1}\|(A_1 + A_2)x\| \leq   \sup_{\|x\| = 1}\{\|A_1x\| + \|A_2x\|\} \leq \sup_{\|x\| = 1}\|A_1x\| + \sup_{\|x\| = 1}\|A_2x\| = \|A_1\| + \|A_2\|
		      \]
		\item Покажем, что $\mathcal{L}(E_1,\,E_2)$ -- полное, если $E_2$ -- полное.

		      Пусть $\{A_n\}_{n = 1}^\infty \subset \mathcal{L}(E_1,\,E_2)$ -- фундаментальна, то есть
		      \[
			      \forall \varepsilon > 0 \: \exists N \: \forall n,\,m \geq N :\: \|A_n - A_m\| < \varepsilon
		      \]
		      Заметим, что
		      \[
			      \forall x \in S_{E_1}(0,\,1) \: \|A_nx - A_mx\| = \|(A_n - A_m)x\| \leq \|A_n - A_m\|\|x\|_{E_1} < \varepsilon
		      \]
		      Получается, что $\forall x \in S_{E_1}(0,\,1)$ последовательность $\{A_nx\}_{n = 1}^\infty$ фундаментальна. Так как $E_2$ -- полное, то эта последовательность сходится. Обозначим её предел через $Ax$. $A$, очевидно, линейный оператор. Покажем, что он ограничен.

		      Так как $\|\cdot\|$ -- непрерывная функция, то $\|A_nx\|_{E_2} \to \|Ax\|_{E_2}$. Воспользуемся тем, что $\{A_n\}_{n = 1}^\infty$ ограничена (из-за фундаментальности), то есть
		      \[
			      \exists K \: \forall n :\: \|A_n\| \leq K \Rightarrow \|A_nx\|_{E_2} \leq \|A_n\|\|x\|_{E_1} \leq K\|x\|_{E_1}
		      \]
		      Переходя к пределу в этом неравенстве, получаем, что $\|Ax\| \leq K\|x\|_{E_1}$. Таким образом, $A$ является ограниченным линейным оператором и лежит в $\mathcal{L}(E_1,\,E_2)$. Осталось показать, что $A_n \to A$.

		      Вспомним фундаментальность:
		      \[
			      \forall x \in S_{E_1}(0,\,1) :\: \|A_nx - A_mx\|_{E_2} < \varepsilon
		      \]
		      Зафиксируем номер $n$ и устремим $m$ к бесконечности. Тогда
		      \[
			      \forall x \in S_{E_1}(0,\,1) :\: \|A_nx - Ax\| \leq \varepsilon
		      \]
		      А значит
		      \[
			      \sup_{\|x\| = 1} \|A_nx - Ax\| \leq \varepsilon
		      \]
		      Это и означает, что $\|A_n - A\| \to 0$, то есть $A_n \to A$.
	\end{enumerate}
\end{proof}

\begin{corollary}
	Если $E$ -- линейное нормированное пространство, то $E^*$ всегда полное.
\end{corollary}

\subsection{Задача о продолжении непрерывного отображения}
\begin{theorem}
	Пусть $E_1$ -- линейное нормированное пространство, $E_2$ -- банахово пространство и $A$ -- линейный ограниченный оператор: $A :\: D(A) \to E_2$, где $D(A)$ -- линейное многообразие в $E_1 = \overline{D(A)}$. Тогда $\exists! \tilde{A} \in \mathcal{L}(E_1,\,E_2)$:
	\begin{enumerate}
		\item $\tilde{A}|_{D(A)} = A$
		\item $\|\tilde{A}\| = \|A\|$
	\end{enumerate}
\end{theorem}

\begin{proof}
	Единственность.

	Пусть есть $\tilde{A}^1,\, \tilde{A}^2$. Из замкнутости $E_1$:
	\[
		\forall x \in E_1 \: \exists\{x_n\}_{n = 1}^\infty \subset D(A) :\: x_n \to x
	\]
	При этом, $\tilde{A}^1x = \lim_{n \to +\infty}Ax_n = \tilde{A}^2x$. Значит, $\tilde{A}^1 = \tilde{A}^2$.

	Существование.

	Определим оператор $\tilde{A}$ по формуле $\tilde{A}x = \lim_{n \to +\infty}Ax_n$. Для коректности необходимо показать:
	\begin{itemize}
		\item Предел $\lim_{n \to +\infty}Ax_n$ существует
		\item Предел не зависит от выбора $\{x_n\}_{n = 1}^\infty$
		\item Действительно $\tilde{A}|_{D(A)} = A$
	\end{itemize}

	Покажем:
	\begin{itemize}
		\item Так как $\{x_n\}_{n = 1}^\infty$ сходится, она фундаментальна, значит последовательность $\{Ax_n\}_{n = 1}^\infty \subset E_2$ фундаментальна. Пространство $E_2$ банахово, поэтому предел $\lim_{n \to +\infty}Ax_n$ существует.
		\item Пусть есть две последовательности $\{x_n'\}_{n=1}^\infty,\, \{x_n''\}_{n=1}^\infty$, сходящиеся к одному и тому же $x \in E_1$, но пусть пределы $\lim_{n \to +\infty} Ax_n',\, \lim_{n \to +\infty}Ax_n''$ разные. Тогда возьмём последовательность $\{y_n\}_{n = 1}^\infty$, полученную чередованием элементов $\{x_n'\}_{n=1}^\infty,\, \{x_n''\}_{n=1}^\infty$. Но получим, что $\{Ay_n\}_{n=1}^\infty$ расходится. Противоречие предыдущему пункту.
		\item Возьмём константную последовательность $\{x\}_{n = 1}^\infty,\, x \in D(A)$. Очевидно, что $\lim_{n \to +\infty} Ax = Ax$. По предыдущему пункту $\lim_{n \to +\infty} Ax_n$ не зависит от выбора последовательности, значит $\tilde{A}|_{D(A)} = A$.
	\end{itemize}
	Осталось показать, что $\tilde{A}$ -- линейный ограниченный оператор. Линейность очевидна из линейности $A$ и предела.

	Ограниченность (а значит и непрерывность) очевидна из непрерывности нормы:
	\[
		\tilde{A}x = \lim_{n \to +\infty}Ax_n \Leftrightarrow \|Ax_n\| \to \|\tilde{A}x\|
	\]
\end{proof}

\subsection{Теорема Банаха-Штейнгауза}
\begin{theorem}
	Теорема Банаха-Штейнгауза.

	Пусть $X,\,Y$ -- линейные нормированные пространства, причём $X$ полно. Пусть $\mathcal{A} \subseteq \mathcal{L}(X,\, Y)$ -- семейство линейных непрерывных операторов. Тогда
	\[
		\forall x \in X :\: \sup \{\|Ax\|_Y \:\vert\: A \in \mathcal{A}\} < +\infty \Rightarrow \sup\{\|A\| \:\vert\: A \in \mathcal{A}\} < +\infty
	\]
	То есть, из поточечной ограниченности следует равномерная ограниченность.
\end{theorem}

\begin{proof}
	Пусть
	\[
		X_n = \{x \in X \:\vert\: \sup_{A \in \mathcal{A}} \|Ax\|_Y \leq n\} = \bigcap_{A \in \mathcal{A}} \{x \in X \:\vert\: \|Ax\|_Y \leq n\}
	\]
	Для любого $A \in \mathcal{A}$ множество $\{x \in X \:\vert\: \|Ax\|_Y \leq n\}$ замкнуто, как прообраз замкнутого шара $\overline{B}(0,\,n) \subset Y$ под действием непрерывного отображения $A$, а пересечение любого количества замкнутых множеств замкнуто.

	Так как $\cup_{n = 1}^\infty X_n = X$, то по теореме Бэра ($X$ полное, а значит все $X_n$ не могут быть не плотными ни в одном шаре, а значит найдётся $m$ и какой-то шар, в котором $X_m$ плотно):
	\[
		\exists m \: \exists \overline{B}(x_0,\,\varepsilon) :\: \overline{B}(x_0,\, \varepsilon) \subseteq X_m
	\]
	Пусть $u \in X,\, \|u\|_X = 1$. Тогда рассмотрим $A \in \mathcal{A}$:
	\begin{align*}
		\|Au\|_Y = \frac{1}{\varepsilon}\|A[\varepsilon u] + Ax_0 - Ax_0\|_Y = \frac{1}{\varepsilon}\|A[x_0 + \varepsilon u] - Ax_0\|_Y \leq \\
		\frac{1}{\varepsilon}\left(\|A[x_0 + \varepsilon u]\|_Y + \|Ax_0\|_Y\right) \leq \frac{1}{\varepsilon}(m + m)
	\end{align*}
	Последнее неравенство верно из-за того, что $x_0 \in X_m,\, x_0 + \varepsilon u \in \overline{B}(x_0,\,\varepsilon) \subset X_m$. В неравенстве сверху можно перейти к супремуму по $u$ и получить, что
	\[
		\forall A \in \mathcal{A} :\: \|A\| \leq \frac{2m}{\varepsilon} < +\infty
	\]
\end{proof}

\subsection{Полнота пространства\dots}
\begin{theorem}
	Полнота $\mathcal{L}(E_1,\, E_2)$ относительно поточечной сходимости.

	Пусть $E_1,\, E_2$ -- банаховы пространства, $\{A_n\}_{n=1}^\infty \subset \mathcal{L}(E_1,\, E_2)$, причём $\forall x \in E_1 :\: \{A_nx\}_{n=1}^\infty$ -- фундаментальная в $E_2$. Тогда
	\[
		\exists A \in \mathcal{L}(E_1,\, E_2) \: \forall x \in E_1 :\: \lim_{n \to +\infty} A_nx = Ax
	\]
\end{theorem}

\begin{proof}
	Так как $\{A_nx\}_{n=1}^\infty$ фундаментальна в банаховом $E_2$, то $\exists \lim_{n \to +\infty} A_nx,\, Ax := \lim_{n \to +\infty}A_nx$.

	Осталось показать, что определённый таким образом оператор -- линейный непрерывный. Линейность очевидна из линейности предела и каждого $A_n$.

	Так как $\forall x \in E_1 :\: \{A_nx\}_{n=1}^\infty$ фундаментальна, то $\{A_n\}_{n=1}^\infty$ поточечно ограничена и по теореме Банаха-Штейнгауза ограничена равномерна:
	\[
		\exists M \: \forall n \in \mathbb{N} :\: \|A_n\| \leq M
	\]
	Тогда
	\[
		\|A_nx\| \leq \|A_n\|\cdot\|x\| \leq M\|x\|,\, \|A_nx\| \to \|Ax\| \Rightarrow \|Ax\| \leq M\|x\|
	\]
	То есть оператор ограничен и непрерывен, а значит $A \in \mathcal{L}(E_1,\, E_2)$.
\end{proof}

\begin{theorem}
	Критерий поточечной сходимости операторов из $\mathcal{L}(E_1,\,E_2)$.

	Пусть $E_1$ -- банахово, $E_2$ -- линейное нормированное пространство. Верно, что $\forall \{A_n\}_{n=1}^\infty \subset \mathcal{L}(E_1,\,E_2)$, $A \in \mathcal{L}(E_1,\,E_2)$:
	\[
		A_n \overset{\text{поточечно}}{\to} A \Leftrightarrow \begin{cases}
			\exists M \: \forall n :\: \|A_n\| \leq M \\
			\forall s \in S,\, E_1 =: \overline{\langle S\rangle}  :\: A_ns \to As
		\end{cases}
	\]
\end{theorem}

\begin{proof}
	($\Rightarrow$) Пункт 2 очевиден. Из поточечной сходимости $\{A_nx\}$ следует поточечная ограниченность $\{A_nx\}_{n=1}^\infty$. Значит, по теорема Банаха-Штейнгауза последовательность $\{\|A_n\|\}_{n=1}^\infty$ ограничена.

	($\Leftarrow$) Так как множество $\langle S\rangle$ всюжу плотно в $E_1$:
	\[
		\forall \varepsilon > 0 \: \forall x \in E_1 \: \exists y \in \langle S\rangle :\: \|x - y\| < \varepsilon
	\]
	Тогда для $y$ (благодаря п.2) верно:
	\[
		\exists N(\varepsilon,\, y) \: \forall n > N :\: \|A_ny - Ay\| < \varepsilon
	\]
	Перейдём к $x \in E_1$:
	\[
		\|A_nx - Ax\| \leq \|Ax - Ay\| + \|A_ny - Ay\| + \|A_ny - A_nx\| \leq \varepsilon\|A\| + \varepsilon + \varepsilon\|A_n\| \leq \varepsilon(\|A\| + M + 1)
	\]
	Значит $A_n$ поточечно сходится к $A$ на $E_1$.
\end{proof}

\section{Сопряжённое пространство\dots}
\subsection{Рисса-Фреше}
\begin{theorem}
	Теорема Рисса-Фреше.

	Пусть существуют гильбертово пространство $H$ и линейный ограниченный функционал $f \in H^*$ в пространстве $H$. Тогда
	\[
		\exists! y \in H \: \forall x \in H :\: f(x) = (y,\,x)
	\]
	При том $\|f\| = \|y\|$.
\end{theorem}

\begin{proof}
	Существование.

	Пусть $f \equiv 0$, тогда $y = 0$. Требуемые свойства, очевидно, выполняются.

	Теперь пусть $f \not\equiv 0$, тогда $\ker f \neq H$, а значит
	\[
		\exists b \in (\ker f)^\bot,\, b \neq 0 :\: f(b) \neq 0
	\]
	Для $x \in H$ рассмотрим $p_x := x - \frac{f(x)}{f(b)}b$. Для него
	\[
		f(p_x) = f(x) - \frac{f(x)}{f(b)}f(b) = 0 \Rightarrow p_x \in \ker f,\, (b,\, p_x) = 0
	\]
	Раскроем последнее выражение:
	\[
		0 = \left(b,\, x - \frac{f(x)}{f(b)}b\right) = (b,\, x) - \frac{f(x)}{f(b)}\|b\|^2 \Rightarrow f(x) = \frac{f(b)}{\|b\|^2}(b,\, x) = \left(\frac{f(b)}{\|b\|^2}b,\, x\right)
	\]
	Обозначим первый член скалярного произведения, как $y$. Как мы видим, он удовлетворяет требованию выразимости из формулировки теоремы.

	Единственность.

	Предположим, что
	\[
		\exists z \: \forall x \in H :\: f(x) = (z,\,x)
	\]
	Тогда, в силу линейности скалярного произведения
	\[
		\forall x \in H :\: (y - z,\, x) = 0
	\]
	Подставим подходящий вектор
	\[
		y - z \in H \Rightarrow 0 = (y - z,\, y -z) = \|y - z\|^2
	\]
	Значит $y - z = 0$ и $z = y$.

	Равенство норм. Из КБШ имеем
	\[
		\forall x \in H :\: \vert f(x)\vert = \vert(y,\, x)\vert \leq \|y\|\cdot\|x\|
	\]
	откуда по определению нормы функционала получаем $\|f\| \leq \|y\|$. Также, так как $y \in H$, то
	\[
		\|y\|^2 = (y,\, y) = f(y) = \vert f(y)\vert \leq \|f\|\cdot\|y\|
	\]
	Значит, $\|y\| \leq \|f\|$. Что в итоге даёт нам $\|y\| = \|f\|$.
\end{proof}

\subsection{Теорема Хана-Банаха и её следствия}
\begin{theorem}
	Теорема Хана-Банаха.

	Пусть $E$ -- ЛНП. $M \subset E$ -- линейное многообразие, $f$ -- линейный ограниченный функционал на $M$. Тогда $\exists \tilde{f} \in E^*$:

	\[
		\begin{cases}
			\tilde{f}|_M = f \\
			\|\tilde{f}\| = \|f\|
		\end{cases}
	\]
\end{theorem}

\begin{note}
	Ниже приведено доказательство для сепарабельных пространств. Теорема верна и для несепарабельных, но там в доказательстве используется трансфинитная идукция.
\end{note}

\begin{proof}
	Будем доопределять $f$: раз $M \neq E \Rightarrow \exists x_0 \not \in M$.

	Тогда рассмотрим $M_1 = M \oplus [x_0]$. Продолжим $f$ на $M_1$ с сохранением нормы:
	\[
		\forall x \in M,\, \alpha \in \mathbb{R} :\: y = x + \alpha x_0 \in M_1 \Rightarrow f_1(y) = f_1(x) + \alpha f_1(x_0) = f_1(x) + \alpha a
	\]
	Существует ли $a$ такое, что $\|f_1\| = \|f\|$?

	Очевидно, что $\|f_1\| \geq \|f\|$, тогда осталось показать существование такого $a$, что $\|f_1\| \leq \|f\|$
	\[
		\exists a \: \forall y \in M_1 :\: \vert f(x) + \alpha a\vert = \vert f_1(y)\vert \leq \|f\|\cdot\|y\| = \|f\|\cdot\|x + \alpha x_0\|
 	\]
	Если $\alpha = 0$, то мы попали на $M$ и неравенство выполнено. Если это не так, то исходное неравенство эквивалентно
	\[
		\left\vert f\left(\frac{x}{\alpha}\right) + a\right\vert \leq \|f\|\cdot\left\|\frac{x}{\alpha} + x_0\right\|
	\]
	Обозначим $z := \frac{x}{\alpha}$. Тогда хочется доказать, что
	\begin{align*}
		-\|f\|\cdot\|z + x_0\| \leq f(z) + a \leq \|f\|\cdot\|z + x_0\| \Leftrightarrow\\
		-f(z) - \|f\|\cdot\|z + x_0\| \leq a \leq -f(z) + \|f\|\cdot\|z + x_0\|
	\end{align*}
	Нам достаточно, чтобы $\sup$ левой части по всем $z$ был меньше $\inf$ правой части неравенства по всем $z$. Тогда на самом деле нам достаточно показать, что
	\[
		\forall z_1,\, z_2 :\: -f(z_1) - \|f\|\cdot\|z_1 + x_0\| \leq -f(z_2) + \|f\|\cdot\|z_2 + x_0\|
	\]
	Преобразуем, следующее должно быть верно:
	\[
		f(z_2) - f(z_1) \leq \|f\|\cdot(\|z_1 + x_0\| + \|z_2 + x_0\|)
	\]
	Это верно, если будет выполнено более сильное утверждение:
	\[
		\vert f(z_2) - f(z_1)\vert \leq \|f\|\cdot(\|z_1 + x_0\| + \|z_2 + x_0\|)
	\]
	Но мы знаем, что 
	\[
		\vert f(z_2) - f(z_1)\vert = \vert f(z_2 - z_1)\vert \leq \|f\|\cdot\|z_2 - z_1\| = \|f\| \cdot\|z_2 + x_0 - x_0 - z_1\| \leq \|f\|\cdot(\|z_1 + x_0\| + \|z_2 + x_0\|)
	\]
	Мы победили, так как это значит, что для любого $z$ действительно можно выбрать такое число $a$.

	Теперь мы готовы запустить индукцию: предположим, что $E$ -- сепарабельное. Тогда пусть 
	\[
		X = \{x_n\}_{n = 0}^\infty :\: \overline{X} = E ;\;\;\;\; M_0 = M,\, x_0 \not\in M_0
	\]
	Берём $x_{n - 1}$ и получаем $M_n = M_{n - 1} + \langle x_{n - 1}\rangle,\, x_n \not\in M_n$. Введём $M_\infty = \cup_{n \in \mathbb{N}} M_n$ -- линейное многообразие.

	Введём
	\[
		f_\infty|_{M_n} := f_n \Rightarrow \overline{X} \subset \overline{M}_\infty
	\]
	По теореме о продолжении $f_\infty$ единственным образом продолжается до $\tilde{f} \in E^*$
\end{proof}

\begin{corollary}
	Пусть $E$ -- линейное нормированное пространство, $M \subsetneq E$ -- линейное многообразие, $x_0 \not\in \overline{M}$. Тогда $\exists f \in E^*$ такой, что 
	\[
		\begin{cases}
			f|_M = 0\\
			f(x_0) = 1\\
			\|f\| = \frac{1}{\rho(x_0,\, M)}
		\end{cases}
	\]
\end{corollary}

\begin{proof}
	Рассмотрим $M_1 = M \oplus \langle x_0\rangle,\, y = x + \alpha x_0$. Тогда
	\[
		f_1(y) = f(x) + \alpha f(x_0) = 0 + \alpha\cdot 1 = \alpha
	\]
	По теореме Хана-Банаха 
	\[
		\exists f \in E^* :\: f|_{M_1} = f_1,\, \|f\|_{E^*} = \|f_1\|_{M_1^*}
	\]
	Осталось показать, что $\|f_1\| = \frac{1}{\rho(x_0,\, M)}$
	\begin{enumerate}
		\item $\forall y \in M_1 :\: y = x + \alpha x_0,\, x \in M,\, \alpha \in \mathbb{K}$. Тогда 
		\[
			\frac{\vert f_1(y)\vert}{\|y\|} = \frac{\vert \alpha\vert}{\|y\|} = \frac{1}{\|\frac{x}{\alpha} + x_0\|} \leq \frac{1}{\rho(x_0,\, M)}
		\]
		\item $\exists$ максимиризирующая последовательность
		\[
			\{x_n\}_{n = 1}^\infty \subset M :\: \|y_n := [x_n + x_0]\| \to \rho
		\]
		Тогда, очевидно
		\[
			\lim_{n \to +\infty} \frac{\vert f_1(y_n)\vert}{\|y_n\|} = \lim_{n \to +\infty}\frac{1}{\|x_n + x_0\|} = \frac{1}{\rho} 
		\]
	\end{enumerate}
\end{proof}

\begin{corollary}
	Если $x \in E$, то $\exists f \in E^*$:
	\[
		\begin{cases}
			\|f\| = 1\\
			f(x) = \|x\|
		\end{cases}
	\]
\end{corollary}

\begin{proof}
	Возьмём для предыдущего следствия $M := \{0\},\, x_0 := \frac{x}{\|x\|}$. Тогда $\exists f$, обладающий всеми нужными свойствами:
	\begin{enumerate}
		\item $f|_{\{0\}} = 0$
		\item $f(x) = \|x\|$
		\item $\|f\| = \frac{1}{\rho(\frac{x}{\|x\|},\, 0)} = 1$
	\end{enumerate}
\end{proof}

\begin{corollary}
	Выполняются следующие утверждения:
	\begin{itemize}
		\item Если $\forall f \in E^* :\: f(x) = f(y)$, то $x = y$.
		\item Если $\forall f \in E^* :\: f(x) = 0$, то $x = 0$
	\end{itemize}
\end{corollary}

\begin{proof}
	Очевидно, что эти два утверждения эквивалентны (можно перенести в одну сторону равенства и воспользоваться линейностью), поэтому будем доказывать только второе.

	От противного: пусть
	\[
		\exists x \neq 0 \: \forall f \in E^* :\: f(x) = 0
	\]
	Но тогда по предыдущему следствию
	\[
		\exists f \in E^* :\: f(x) = \|x\| \neq 0
	\]
	Противоречие!
\end{proof}

\begin{corollary}
	$\forall x \in E$ его норма может быть выражена, как
	\[
		\|x\| = \sup_{f \in E^*,\, \|f\|_{E^*} = 1} \vert f(x)\vert
	\]
\end{corollary}

\begin{proof}
	Оценка сверху очевидна:
	\[
		\sup_{f \in E^*,\, \|f\|_{E^*} = 1}\vert f(x)\vert \leq \sup_{f \in E^*,\, \|f\|_{E^*} = 1} \|f\|\cdot\|x\| = \|x\|
	\]
	Для неравенства в другую сторону зафиксируем $x$. Если $x = 0$, то всё тривиально. Иначе
	\[
		\sup_{f \in E^*,\, \|f\|_{E^*} = 1} \vert f(x)\vert \geq \vert f_x(x)\vert = \|x\|
	\]
	где $f_x(x)$ -- функция из одного из предыдущих следствий.
\end{proof}

\end{document}